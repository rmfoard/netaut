%%%%%%%%%%%%%%%%%%%%%%%%%%%%%%%%%%%%%%%%%
% This file was derived from:
% Journal Article
% LaTeX Template
% Version 1.4 (15/5/16)
%
% This template has been downloaded from:
% http://www.LaTeXTemplates.com
%
% Original author:
% Frits Wenneker (http://www.howtotex.com) with extensive modifications by
% Vel (vel@LaTeXTemplates.com)
%
% License:
% CC BY-NC-SA 3.0 (http://creativecommons.org/licenses/by-nc-sa/3.0/)
%
%%%%%%%%%%%%%%%%%%%%%%%%%%%%%%%%%%%%%%%%%

%----------------------------------------------------------------------------------------
%	PACKAGES AND OTHER DOCUMENT CONFIGURATIONS
%----------------------------------------------------------------------------------------

\documentclass[twoside,twocolumn]{article}

\usepackage{blindtext} % Package to generate dummy text throughout this template 

\usepackage[sc]{mathpazo} % Use the Palatino font
\usepackage[T1]{fontenc} % Use 8-bit encoding that has 256 glyphs
\linespread{1.05} % Line spacing - Palatino needs more space between lines
\usepackage{microtype} % Slightly tweak font spacing for aesthetics

\usepackage[english]{babel} % Language hyphenation and typographical rules

\usepackage[hmarginratio=1:1,top=32mm,columnsep=20pt]{geometry} % Document margins
\usepackage[hang, small,labelfont=bf,up,textfont=it,up]{caption} % Custom captions under/above floats in tables or figures
\usepackage{booktabs} % Horizontal rules in tables

\usepackage{lettrine} % The lettrine is the first enlarged letter at the beginning of the text

\usepackage{enumitem} % Customized lists
\setlist[itemize]{noitemsep} % Make itemize lists more compact

\usepackage{abstract} % Allows abstract customization
\renewcommand{\abstractnamefont}{\normalfont\bfseries} % Set the "Abstract" text to bold
\renewcommand{\abstracttextfont}{\normalfont\small\itshape} % Set the abstract itself to small italic text

\usepackage{titlesec} % Allows customization of titles
\renewcommand\thesection{\Roman{section}} % Roman numerals for the sections
\renewcommand\thesubsection{\roman{subsection}} % roman numerals for subsections
\titleformat{\section}[block]{\large\scshape\centering}{\thesection.}{1em}{} % Change the look of the section titles
\titleformat{\subsection}[block]{\large}{\thesubsection.}{1em}{} % Change the look of the section titles

\usepackage{fancyhdr} % Headers and footers
\pagestyle{fancy} % All pages have headers and footers
\fancyhead{} % Blank out the default header
\fancyfoot{} % Blank out the default footer
\fancyhead[C]{Running title $\bullet$ May 2016 $\bullet$ Vol. XXI, No. 1} % Custom header text
\fancyfoot[RO,LE]{\thepage} % Custom footer text

\usepackage{titling} % Customizing the title section

\usepackage{hyperref} % For hyperlinks in the PDF

%----------------------------------------------------------------------------------------
%	TITLE SECTION
%----------------------------------------------------------------------------------------

\setlength{\droptitle}{-4\baselineskip} % Move the title up

\pretitle{\begin{center}\Huge\bfseries} % Article title formatting
\posttitle{\end{center}} % Article title closing formatting
\title{Emergent Behavior in Graph Automata} % Article title
\author{%
\textsc{Richard Foard}\thanks{A thank you or further information} \\[1ex] % Your name
\normalsize Georgia Tech Research Institute \\ % Your institution
\normalsize \href{mailto:richard.foard@gtri.gatech.edu}{richard.foard@gtri.gatech.edu} % Your email address
}
\date{\today} % Leave empty to omit a date
\renewcommand{\maketitlehookd}{%
\begin{abstract}
\noindent A simple, rule-based
graph automaton was defined and simulated. Each of its possible rules specifies a set
of local topology and node state changes to be applied iteratively, starting with a random initial graph.
Simulation runs were performed using many different rules, each run terminating when its graph
collapsed or cycled.
Evolving and terminal graph states were analyzed macroscopically, using
aggregate statistics, and microscopically, by inspecting graph structures.
Results were compared with those from simulations of a machine that
iterated by applying random, rather than rule-based, changes. We found that...
\end{abstract}
}

%----------------------------------------------------------------------------------------

\begin{document}

% Print the title
\maketitle

%----------------------------------------------------------------------------------------
%	ARTICLE CONTENTS
%----------------------------------------------------------------------------------------

\section{Introduction}

\lettrine[nindent=0em,lines=3]{S} ince Alan Turing conceived his
universal machine in 1936, simple abstract automata have been studied. Interest broadened
beyond the theoretical realm in 1970,
when Conway published his \textit{Game of Life} simulations that highlighted the ability
of uncomplicated cellular automata to behave in complex ways. Researchers in the
nascent field of complexity theory also devoted attention to similar phenomena, such as the
"sandpile" first explored by Per Bak (true??).

In \textit{A New Kind of Science} (199?), Stephen Wolfram methodically analyzed
a variety of cellular automata types. His voluminous results cited particular inspiration from the
behavior of a one-dimensional machine running "Rule 110." Among his observations were... (list
irreducibility, etc.)

Wolfram and others have also suggested that some natural processes that were
previously thought to evolve by natural selection are instead
manifestations of things that nature found "easy" to accomplish using
the same fundamental principles that underlie simple automata.

In this work, we analyze a graph automaton -- a machine that operates on the same principles as
cellular automata, but runs using a graph rather than a "tape," or grid, as a substrate.
Where cellular machines define cell adjacency spatially, our machines use graph topology.

%------------------------------------------------

\section{Methods}

A rule-based automaton was designed and implemented in simulation. The simulator was run repeatedly, using rules
selected at random from the large universe of possibilities. Measurements were recorded as the graphs
evolved through each run. The resulting database accumulated data on many thousands of runs. The body of data
was analyzed macroscopically, using aggregate
statistics, and microscopically, by inspecting statistical and graphic "snapshots" from individual runs.

Each simulation run begins with a randomly generated graph and progresses by
iteratively modifying its edges and node states; changes are made according to a rule selected
for the run. Each rule, encoded as a large integer, is effectively
a simple program  that specifies, based on the state
of each node and its two topological neighbors, how local changes should be applied.

A simulation run proceeds as follows: (describe initial condition and iteration procedure)

\subsection{Graphs}

Nodes may be in one of two states: black or white.
Graph structure is highly restricted in order to simplify simulation and analysis:

\begin{itemize}
\item Edges are directed.
\item All nodes have out-degree of exactly two.
\item Multiple, like-directed edges between two nodes are prohibited.
\item "Self-linking" edges are permitted.
\end{itemize}

These conditions are established when creating initial graphs and maintained
throughout execution by "pruning," as described in \textit{Maintaining Structure}.

An initial, random graph of size N is constructed by assigning two edges to each node, with
the destination of each selected at random from all nodes, observing the
constraint that both edges cannot have the same destination node. As a consequence,
all nodes have out-degree 2; in-degree can vary from 0 to 2N (wrong!). Each node is assigned
a black or white state, with equal likelihood except where otherwise noted.

\subsection{Rules}

The rules for Machine C. Random rules for comparison are not rules \textit{per se}.

\subsection{Maintaining Structure}

Invariants are maintained in all simulations -- both for the rule-based and
the random machine. <How the invariants are maintained.>

\subsection{The Simulator}

Input parameters and statistics gathered. Data base and analysis tooling.

\subsection{Run Selection}

There are 722 trillion (<formula>) possible rules. Making choices amidst
many degrees of freedom. Varying initial graph size. Rule-based versus
random.

Text requiring further explanation\footnote{Example footnote}.

%------------------------------------------------

\section{Results}

Because initial graphs are finite, and graph size is either maintained
or reduced in each iteration, all runs must terminate with either
a collapsing or cycling graph. The immense number of possible graphs
(formula), however, precludes carrying every run to completion, so a
third outcome, "unfinished," must be considered.

Overview commentary. How results from random Machine R differ.

\subsection{Unhappy Endings -- Cycles and Collapses}

All runs end in one of three ways... Graphs are finite. They begin
with n nodes and 2n edges and can only remain at theit initial
size or grow smaller as iterations proceed.

For this one, we'll have to dust off the old self-organizing criticality
references. Are collapses like sandpile avalanches?

\subsection{In the Long Run...}

\subsection{Self-Organization -- Made to Order?}

\subsection{Learning the Hard Way}

\subsection{Analysis -- Doing the math}

\subsection{Evolution by Design}

\subsection{Interesting Configurations -- Photo Ops}

\begin{table}
\caption{Example table}
\centering
\begin{tabular}{llr}
\toprule
\multicolumn{2}{c}{Name} \\
\cmidrule(r){1-2}
First name & Last Name & Grade \\
\midrule
John & Doe & $7.5$ \\
Richard & Miles & $2$ \\
\bottomrule
\end{tabular}
\end{table}

Some dummy text was here.

\begin{equation}
\label{eq:emc}
e = mc^2
\end{equation}

Some dummy text was here.

%------------------------------------------------

\section{Discussion}

\subsection{Subsection One}

A statement requiring citation \cite{Figueredo:2009dg}.
Some dummy text was here, too.

\subsection{Subsection Two}

Some dummy text was here, too.

%----------------------------------------------------------------------------------------
%	REFERENCE LIST
%----------------------------------------------------------------------------------------

\begin{thebibliography}{99} % Bibliography - this is intentionally simple in this template

\bibitem[Figueredo and Wolf, 2009]{Figueredo:2009dg}
Figueredo, A.~J. and Wolf, P. S.~A. (2009).
\newblock Assortative pairing and life history strategy - a cross-cultural
  study.
\newblock {\em Human Nature}, 20:317--330.
 
\end{thebibliography}

%----------------------------------------------------------------------------------------

\end{document}
